% Options for packages loaded elsewhere
\PassOptionsToPackage{unicode}{hyperref}
\PassOptionsToPackage{hyphens}{url}
\PassOptionsToPackage{dvipsnames,svgnames,x11names}{xcolor}
%
\documentclass[
  letterpaper,
  DIV=11,
  numbers=noendperiod]{scrartcl}

\usepackage{amsmath,amssymb}
\usepackage{lmodern}
\usepackage{iftex}
\ifPDFTeX
  \usepackage[T1]{fontenc}
  \usepackage[utf8]{inputenc}
  \usepackage{textcomp} % provide euro and other symbols
\else % if luatex or xetex
  \usepackage{unicode-math}
  \defaultfontfeatures{Scale=MatchLowercase}
  \defaultfontfeatures[\rmfamily]{Ligatures=TeX,Scale=1}
\fi
% Use upquote if available, for straight quotes in verbatim environments
\IfFileExists{upquote.sty}{\usepackage{upquote}}{}
\IfFileExists{microtype.sty}{% use microtype if available
  \usepackage[]{microtype}
  \UseMicrotypeSet[protrusion]{basicmath} % disable protrusion for tt fonts
}{}
\makeatletter
\@ifundefined{KOMAClassName}{% if non-KOMA class
  \IfFileExists{parskip.sty}{%
    \usepackage{parskip}
  }{% else
    \setlength{\parindent}{0pt}
    \setlength{\parskip}{6pt plus 2pt minus 1pt}}
}{% if KOMA class
  \KOMAoptions{parskip=half}}
\makeatother
\usepackage{xcolor}
\setlength{\emergencystretch}{3em} % prevent overfull lines
\setcounter{secnumdepth}{-\maxdimen} % remove section numbering
% Make \paragraph and \subparagraph free-standing
\ifx\paragraph\undefined\else
  \let\oldparagraph\paragraph
  \renewcommand{\paragraph}[1]{\oldparagraph{#1}\mbox{}}
\fi
\ifx\subparagraph\undefined\else
  \let\oldsubparagraph\subparagraph
  \renewcommand{\subparagraph}[1]{\oldsubparagraph{#1}\mbox{}}
\fi


\providecommand{\tightlist}{%
  \setlength{\itemsep}{0pt}\setlength{\parskip}{0pt}}\usepackage{longtable,booktabs,array}
\usepackage{calc} % for calculating minipage widths
% Correct order of tables after \paragraph or \subparagraph
\usepackage{etoolbox}
\makeatletter
\patchcmd\longtable{\par}{\if@noskipsec\mbox{}\fi\par}{}{}
\makeatother
% Allow footnotes in longtable head/foot
\IfFileExists{footnotehyper.sty}{\usepackage{footnotehyper}}{\usepackage{footnote}}
\makesavenoteenv{longtable}
\usepackage{graphicx}
\makeatletter
\def\maxwidth{\ifdim\Gin@nat@width>\linewidth\linewidth\else\Gin@nat@width\fi}
\def\maxheight{\ifdim\Gin@nat@height>\textheight\textheight\else\Gin@nat@height\fi}
\makeatother
% Scale images if necessary, so that they will not overflow the page
% margins by default, and it is still possible to overwrite the defaults
% using explicit options in \includegraphics[width, height, ...]{}
\setkeys{Gin}{width=\maxwidth,height=\maxheight,keepaspectratio}
% Set default figure placement to htbp
\makeatletter
\def\fps@figure{htbp}
\makeatother

\KOMAoption{captions}{tableheading}
\makeatletter
\makeatother
\makeatletter
\makeatother
\makeatletter
\@ifpackageloaded{caption}{}{\usepackage{caption}}
\AtBeginDocument{%
\ifdefined\contentsname
  \renewcommand*\contentsname{Table of contents}
\else
  \newcommand\contentsname{Table of contents}
\fi
\ifdefined\listfigurename
  \renewcommand*\listfigurename{List of Figures}
\else
  \newcommand\listfigurename{List of Figures}
\fi
\ifdefined\listtablename
  \renewcommand*\listtablename{List of Tables}
\else
  \newcommand\listtablename{List of Tables}
\fi
\ifdefined\figurename
  \renewcommand*\figurename{Figure}
\else
  \newcommand\figurename{Figure}
\fi
\ifdefined\tablename
  \renewcommand*\tablename{Table}
\else
  \newcommand\tablename{Table}
\fi
}
\@ifpackageloaded{float}{}{\usepackage{float}}
\floatstyle{ruled}
\@ifundefined{c@chapter}{\newfloat{codelisting}{h}{lop}}{\newfloat{codelisting}{h}{lop}[chapter]}
\floatname{codelisting}{Listing}
\newcommand*\listoflistings{\listof{codelisting}{List of Listings}}
\makeatother
\makeatletter
\@ifpackageloaded{caption}{}{\usepackage{caption}}
\@ifpackageloaded{subcaption}{}{\usepackage{subcaption}}
\makeatother
\makeatletter
\@ifpackageloaded{tcolorbox}{}{\usepackage[many]{tcolorbox}}
\makeatother
\makeatletter
\@ifundefined{shadecolor}{\definecolor{shadecolor}{rgb}{.97, .97, .97}}
\makeatother
\makeatletter
\makeatother
\ifLuaTeX
  \usepackage{selnolig}  % disable illegal ligatures
\fi
\IfFileExists{bookmark.sty}{\usepackage{bookmark}}{\usepackage{hyperref}}
\IfFileExists{xurl.sty}{\usepackage{xurl}}{} % add URL line breaks if available
\urlstyle{same} % disable monospaced font for URLs
\hypersetup{
  colorlinks=true,
  linkcolor={blue},
  filecolor={Maroon},
  citecolor={Blue},
  urlcolor={Blue},
  pdfcreator={LaTeX via pandoc}}

\author{}
\date{}

\begin{document}
\ifdefined\Shaded\renewenvironment{Shaded}{\begin{tcolorbox}[frame hidden, enhanced, boxrule=0pt, breakable, borderline west={3pt}{0pt}{shadecolor}, interior hidden, sharp corners]}{\end{tcolorbox}}\fi

\hypertarget{salari-ed-inquadramenti}{%
\section{Salari ed Inquadramenti}\label{salari-ed-inquadramenti}}

\hypertarget{section}{%
\subsection{}\label{section}}

\begin{longtable}[]{@{}
  >{\raggedright\arraybackslash}p{(\columnwidth - 4\tabcolsep) * \real{0.4615}}
  >{\raggedright\arraybackslash}p{(\columnwidth - 4\tabcolsep) * \real{0.2821}}
  >{\raggedright\arraybackslash}p{(\columnwidth - 4\tabcolsep) * \real{0.2564}}@{}}
\toprule()
\begin{minipage}[b]{\linewidth}\raggedright
Legge di Interesse
\end{minipage} & \begin{minipage}[b]{\linewidth}\raggedright
Descrizione
\end{minipage} & \begin{minipage}[b]{\linewidth}\raggedright
Situazione
\end{minipage} \\
\midrule()
\endhead
art. 22, comma 15 dlgs 75/2017 & valorizzazione professionalità interne
& 582 domande 105 posti \textbf{Sottoinquadramento} \\
ex art. 15 & carriera & formalmente approvato scorrimento circa 800
persone. \textbf{Molti in pensioni o prossimi.} \\
art 20 comma2 dlgs 75/2017, comma 669 legge 205/2017, Direttiva 99/70/CE
& Riconoscimento anzianità TD, Assegnisti, Borsisti & In predisposizione
ricorso collettivo USB EPR. \textbf{132 CNR} \\
\bottomrule()
\end{longtable}

\hypertarget{section-1}{%
\subsection{}\label{section-1}}

\begin{itemize}
\item
  \textbf{PNNR2} : dimezzamento assegni di ricerca erogabili, il resto
  diventano borse oppure spostare il personale sulla ricerca privata
  finanziata
\item
  \textbf{IMPATTO}: demansionamento e perdita di salario, perdita di
  personale qualificato
\item
  \textbf{TD} dai fondi europei che fine fanno?
\item
  \textbf{DISTORSIONE} missione del CNR e del concetto di ricerca
  pubblica.
\item
  Non è solo ricerca e sviluppo industriale né ricerca di base
\item
  \href{https://www.cnr.it/sites/default/files/public/media/amministrazione_trasparente/Provvedimento_93-2018.pdf}{Dallo
  statuto CNR, art 3.}:

  \begin{itemize}
  \item
    fornisce attività di consulenza, certificazione e supporto
    tecnico-scientifico agli organi costituzionali e alle
    amministrazioni pubbliche
  \item
    promuove l'applicazione della Carta europea dei ricercatori, del
    codice di condotta per il reclutamento dei ricercatori e delle
    azioni europee per la definizione di un quadro di riferimento per le
    carriere nello spazio europeo della ricerca;
  \item
    promuove la diffusione della conoscenza nella società;
  \end{itemize}
\end{itemize}

\hypertarget{piano-di-rilancio-cnr}{%
\section{Piano di Rilancio CNR}\label{piano-di-rilancio-cnr}}

\hypertarget{section-2}{%
\subsection{}\label{section-2}}

\begin{itemize}
\item
  Non trasparente.
\item
  Un percorso partecipato \textbf{non è} scegliere il nome di una nave o
  un logo
\item
  Diversi Istituti CNR hanno competenze ad hoc
  \href{https://www.istc.cnr.it/en/content/manuale-di-progettazione-partecipata-con-i-bambini-e-le-bambine}{ISTC},
  \href{https://www.issirfa.cnr.it/attivita-di-ricerca/progetti-conclusi/progetto-percorsi/}{ISSIFRA}
\item
  \textbf{Voci}: bocciatura dai revisori dei conti in particolare il
  revisore MEF
\item
  Meccanismi peggiorativi su reclutamento sulla falsa riga di PNNR2
\item
  Amministrazione su scala regionale fuori dagli istituti:

  \begin{itemize}
  \tightlist
  \item
    Molti istituti sono in più regioni
  \item
    Livello intermedio che potrebbe dare interpretazioni diverse
  \item
    Es. IAC Roma (Lazio-Toscana) vs IAC Bari (Puglia-Basilicata)
  \end{itemize}
\end{itemize}



\end{document}
